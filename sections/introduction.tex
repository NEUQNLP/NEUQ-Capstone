\section{绪论}
  	\subsection{课题的背景和意义}
	
	即时通讯(Instant Messaging,简称 IM)这种通讯手段已经融入生活的各个方面,随着近年来各种移动 IM 应用的流行,即时通讯已经成为人与人之间交流的重要工具。尤其是近几年的快速发展,即时通讯的功能也日渐丰富,由最初的简单文字聊天逐渐扩展到图片、语音、视频等多种形式,成为集交流、资讯、娱乐、办公协作等一体的综合化信息平台。
	
	省略一段文字...
	
	
	\subsection{国内外即时通讯的发展状况}
	
	由于即时通讯软件的飞速发展和其特有的实时性、扩平台性、效率高等诸多优势,使之成为人们最喜爱的网络沟通手段之一。在移动互联网的范畴内,国内外涌现出大量的即时通讯软件,国内以腾讯的QQ、微信最受欢迎,国外最著名的当属 WhatsApp,即时通讯技术在手机端展现出强大的活力。
	
	省略一段文字...
	
  	\subsection{课题研究的主要方法及内容}

 	本课题主要工作是...

  	本课题主要包含以下几个方面内容:
  
  	\begin{enumerate}
  
    \item 调研主流即时通讯软件的功能...
    
    \item 深入研究...
    
    \item 设计实现...

    \item 实现整个...
    
  	\end{enumerate}

  	\subsection{论文组织结构}
  
  	本文主要围绕相关技术选型,需求分析,系统整体设计、详细设计,部署与测试等方面来进行论述,共分为6章,各章内容如下:
	
	第1章...
	
    第2章...
    
    第3章...
    
    第4章... 
       
    第5章...
    
    第6章...
    
    为了更好的理解 ...
    
\clearpage